\pdfminorversion=4
\documentclass[]{article}
\usepackage[utf8]{inputenc}
\usepackage{amssymb,latexsym,amsmath}
\usepackage[a4paper,top=3cm,bottom=2cm,left=3cm,right=3cm,marginparwidth=1.75cm]{geometry}
\usepackage{graphicx}
\usepackage[colorlinks=true, allcolors=blue]{hyperref}
\begin{document}

\begin{itemize}
  \item Let \( O_i \) represent the i-th operational data.
  \item Let \( M_j \) represent the j-th mission.
  \item Let \( C(M_j) \) represent the set of child missions of mission \( M_j \).
  \item Let \( u(O_i, M_j) \) represent the utilization score of operational data \( O_i \) for mission \( M_j \), where the score is 1 if \( O_i \) is used by \( M_j \), and 0 if not.
\end{itemize}

For leaf missions, the utilization score is directly assigned:

\[ u(O_i, M_j) = 1 \quad \text{if operational data } O_i \text{ is used by leaf mission } M_j \]

For non-leaf missions, the utilization score for each operational data is the average of the scores of that data across all child missions:

\[ u(O_i, M_j) = \frac{1}{|C(M_j)|} \sum_{k \in C(M_j)} u(O_i, M_k) \]

Where \( |C(M_j)| \) is the number of child missions of \( M_j \).

The final score of operational data \( O_i \) for mission \( M_j \) is then normalized if necessary, to ensure that the score is within a specific range (e.g., 0 to 100\%). The normalization can be represented as:

\[ S(O_i, M_j) = \left( \frac{u(O_i, M_j) - \min(u)}{\max(u) - \min(u)} \right) \times 100 \% \]

where \( S(O_i, M_j) \) is the final normalized score of operational data \( O_i \) for mission \( M_j \), and \( \min(u) \) and \( \max(u) \) are the minimum and maximum utilization scores across all operational data and missions before normalization.

\end{document}
